%
%  Abstract
%

\begin{abstract}
\addcontentsline{toc}{chapter}{Abstract}
%todo: max 350 words

This dissertation describes a search for $R$-parity violating supersymmetry, motivated by the stringent limits set on $R$-parity conserving models from Run I and Run II of the LHC.
These limits have excluded gluno masses up to approximately 2~\TeV in mass, which is the rough scale expected for supersymmetry to ``naturally'' solve the Hierarchy Problem.
These constraints, however, can be evaded by considering $R$-parity violating models, in which the lightest supersymmetric particle can decay to Standard Model particles and does not produce a large missing transverse momentum signature.

To avoid conflicts with experimental measurements, such as proton decay, the framework of Minimal Flavor Violation is applied, resulting in the largest $R$-parity violating coupling being between a top, bottom, and strange quark.
Therefore, this search uses the pair production of gluinos that decay via \rpvDecay as a benchmark model and generically looks for new physics with a signature of a single lepton, large jet and bottom quark jet multiplicities, and high sum of large-radius jet masses, without any requirement on the missing transverse momentum in an event.

The search is conducted with $35.9~\ifb$ of $\sqrt{s} = 13~\TeV$ proton-proton collisions collected by the CMS experiment in 2016.
The background is estimated through a maximum-likelihood fit of the \Nb distribution across bins of jet multiplicity and sum of large-radius jet masses.
No evidence of new physics is observed, and limits on a simplified model, in which gluinos decay promptly via \smsDecay, are set, excluding gluino masses below 1610~\GeV at the 95\% confidence level.

%\abstractsignature
\end{abstract}


