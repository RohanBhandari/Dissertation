\chapter{Summary and Conclusions}

This dissertation has presented a search for new physics in $35.9~\ifb$ of data produced by $\sqrt{s} = 13~\TeV$ proton-proton collisions from the LHC and collected by the CMS detector in 2016. 
The search investigates events with a final state of a single lepton, large jet and b-tagged jet multiplicities, and high sum of large-radius jet masses.
This final state is motivated by a \RP and minimal-flavor violating supersymmetric model, in which gluinos are pair produced and decay via \rpvDecay.
This search, however, is structured to be generically sensitive to models with high-mass signatures and many b-tagged jets, while the lack of an explicit \MET requirement increases the search coverage to even \RPC models in which there is little \MET produced.

The background is predicted through a global maximum-likelihood fit of the distribution of number of b-tagged jets across bins of number of jets and sum of masses of large-radius jets.
The normalizations of the dominant backgrounds are measured in data, while their shapes are taken from simulation with corrections measured in data control samples and are allowed to vary according to their uncertainties.

The main uncertainty in the background prediction method is the statistical uncertainty due to the data sample size, while the largest systematic uncertainties arise from the modeling of the gluon splitting rate and the b quark tagging efficiency and mistag rate.

Results from the background-only fit found the observed data to be well modelled and consistent with the background-only hypothesis.
Accordingly, limits are set on a benchmark simplified model where pair produced gluinos each decay via \smsDecay.
An upper limit of approximately $10~\fb$ is measured for the pair production of gluinos in this scenario, which corresponds to excluding gluino masses below $1610~\GeV$ at a 95\% confidence level.

These limits represent a significant improvement on the coverage of R-parity violating supersymmetric models, improving on results obtained at $\sqrt{s} = 8~\TeV$~\cite{Khachatryan:2016iqn,Aad:2015lea} by approximately $600~\GeV$, and are among the most stringent limits set by both the CMS and ATLAS Collaborations~\cite{Aaboud:2017faq,Aaboud:2018lpl}.

The original hopes for Run II of the LHC were for a quick discovery of new physics, after which these new particles could be studied in detail and solutions to universal questions obtained.
Of course, this has not been the case, and instead many limits have been set.
The idea of naturalness, however, remains a highly-compelling reason for new physics to be just around the corner, and despite the lack of evidence of such a thing, there is still significant phase space left in which (\RPV) supersymmetric models may be hiding, while still meeting the naturalnes guidelines outlined in Section~\ref{sec:susy_constraints}.
Thus, supersymmetric models are an important class of signatures for which searches need to be continued, and, if naturalness is truly believed in, comfort can be taken in that every null result is one step closer to a discovery.

I am proud to know that the contents of this dissertation represent my contribution of one such step to the corpus of particle physics, and I sincerely hope that the next generation of SUSY-searchers find themselves describing their results not in terms of limits but of significances.
Good luck and happy hunting!
