\chapter{Introduction}

The discovery of the Higgs boson at the end of Run I of the Large Hadron Collider marked a transition in high energy physics from asking questions like ``At what mass will we find the Higgs boson?'' to ``Something needs to be out there, but what?''.
The scope of these questions reflects the different strategies of their respective search programs.
For the Higgs bosons, its couplings and branching fractions as a function of mass were already well described by the Standard Model, and from these expectations, one could formulate a targeted, multi-channel approach for discovery.
On the other hand, search strategies are now driven by motivated but still theoretical models and the goal is to cast as wide a net as possible in the hopes of finding hints of something.

To this end, an incredible amount of work has been done over the last few years to cover great amounts of model space.
These initial searches rightfully worked under the principle of Occam's razor and searched for the simplest models that could provide solutions to the most problems.
Unfortunately, no evidence of new physics has been observed, and it appears that nature is not that kind and likely won't take the simple forms hoped for.
However, if one is willing to make sacrifices, in the form of increasing model complexity or giving up potential solutions, there is still much more intriguing phase space that needs to be covered. 

This dissertation describes one such search using a dataset corresponding to $35.9~\ifb$ of $\sqrt{s} = 13~\TeV$ proton-proton collisions collected in 2016. 
A publication briefly describing the search and its results has been been submitted for publication, Reference~\cite{Sirunyan:2017dhe}, with additional information presented in Reference~\cite{SUS-16-040_supp}.
Preliminary results using $2.3~\ifb$ are given in Reference~\cite{CMS-PAS-SUS-16-013}, while results using similar methods at $8~\TeV$ were published in Reference~\cite{Khachatryan:2016iqn}.

In particular, Chapter 2 discusses the current state of the Standard Model and its deficiencies, introducing the Hierarchy Problem.
Chapter 3 then motivates minimal flavor violating supersymmetry, a class of models worth experimental attention in which the constraints on ``natural'' solutions to the Hierarchy Problem are reduced in exchange for disfavoring explanations for dark matter.
Chapters 4 and 5 describe the Large Hadron Collider and Compact Muon Solenoid detector used for producing and collecting the relevant data samples, as well as how particles are identified for reconstructing collision events.
Chapter 6 presents in more detail the data sample along with the generation of the simulated samples used.

At this point discussion of the search strategy begins with Chapter 7 presenting the event selection that defines the analysis region.
Chapter 8 gives an overview of the maximum-likelihood fit used for predicting the background, while the systematic uncertainties that are assessed for this procedure are provided in Chapter 9.
The technical aspects of the likelihood model along with extensive validation tests are detailed in Chapter 10.
Finally, the results from this search are presented in Chapter 11, with a summary and conclusions given in Chapter 12.
